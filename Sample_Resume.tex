%------------------------ RESUME TEMPLATE -------------------------------------

%------------------------------ PREAMBLE --------------------------------------

% You probably won't need to change any of this. It might be helpful looking at the comments to get an idea of how sections are formatted. Otherwise, scroll down to HEADER to start editing content!

\documentclass{resume2} % Use the custom resume2.cls style
\usepackage[autostyle]{csquotes}
\usepackage{amsfonts}
\usepackage{multicol}
\usepackage{hyperref}
\setlength{\columnsep}{1cm}

\newcommand\textbox[1]{%
  \parbox{.48\textwidth}{#1}%
}

\newenvironment{rEmptySubsection}[3]{
  %%%%%%%%%%%%%%%%%%%%%% Default Layout: %%%%%%%%%%%%%%%%%%%%%%%%
  %%    Employer (bold)                     Dates (regular)    %%
  %%    Title (emphasis)                Location (emphasis)    %%
  %%%%%%%%%%%%%%%%%%%%%%%%%%%%%%%%%%%%%%%%%%%%%%%%%%%%%%%%%%%%%%%
  {\bf #1}                 \hfill                  {    #2}% Stop a space
  \ifthenelse{\equal{#3}{}}{}{
  \\
  {\em #3}                                 % Stop a space
  }
  % empty
}{
}

\newenvironment{rNSubsection}[4]{
  %%%%%%%%%%%%%%%%%%%%%% Default Layout: %%%%%%%%%%%%%%%%%%%%%%%%
  %%    Employer (bold)                     Dates (regular)    %%
  %%    Title (emphasis)                Location (emphasis)    %%
  %%%%%%%%%%%%%%%%%%%%%%%%%%%%%%%%%%%%%%%%%%%%%%%%%%%%%%%%%%%%%%%
  {\bf #1}                 \hfill                  {    #2}% Stop a space
  \ifthenelse{\equal{#3}{}}{}{
  \\
  {\em #3}                 \hfill                  {\em #4}% Stop a space
  }\smallskip
  % \cdot used for bullets, items non-indented
  \begin{list}{}{\leftmargin=0em}
  \itemsep -0.5em \vspace{-0.5em}
}{
  \end{list}
  \vspace{0.5em}
}

\newenvironment{rMSubsection}[3]{
  %%%%%%%%%%%%%%%%%%%%%% Default Layout: %%%%%%%%%%%%%%%%%%%%%%%%
  %%    Employer (bold)                     Dates (regular)    %%
  %%    Title (emphasis)                Location (emphasis)    %%
  %%%%%%%%%%%%%%%%%%%%%%%%%%%%%%%%%%%%%%%%%%%%%%%%%%%%%%%%%%%%%%%
  {\bf #1}                 \hfill                  {    #2}% Stop a space
  \ifthenelse{\equal{#3}{}}{}{
  \\
  { #3}                                 % Stop a space
  }
  % empty
}{
}

\newcommand{\tab}[1]{\hspace{.2667\textwidth}\rlap{#1}}
\newcommand{\itab}[1]{\hspace{0em}\rlap{#1}}


%------------------------------ HEADER -----------------------------------------

\name{Giulio Centorame}
\address{Phone \\ Email Address \\ http://giuliocentora.me/}
\address{}
\address{Address 1 \\  Address 2}

\begin{document}

\thispagestyle{plain}

%------------------------------ CONTENT ---------------------------------------

%----------------------------------------------------------------------------------------

\begin{rSection}{Education}

% \textbox{\textbf{Degree Title and Discipline}\hfill}\textbox{\hfill Graduation Date}\\
% \textbox{University Name, \textit{University Location}}{\hfill GPA (if applicable)}

% If your degree title/discipline spills onto the next line, put it in \mbox{}.

\textbox{\textbf{MSc, Genes, environment and development in psychology and psychiatry}\hfill}\textbox{\hfill September 2019}\\
\textbox{King's College London, \textit{London, UK}\hfill}{\hfill Merit}

\textbox{\textbf{Bachelor of Science, Psychology}\hfill}\textbox{\hfill June 2018}\\
\textbox{Sigmund Freud University, \textit{Vienna, Austria \& Milan, Italy}}{\hfill Grade: 1.08}


\end{rSection}

%----------------------------------------------------------------------------------------

\begin{rSection}{Professional and Research Experience}

% rSubsection works well if you want to have a job position that includes several bullet points beneath it.

% \begin{rSubsection}{Job Title}{Dates - Dates}{Company/Organization}{Location}
% \item Skills
% \item Skills
% \end{rSubsection}

\begin{rSubsection}{Research Assistant}{August 2017---June 2018}{Sigmund Freud University, Milan}{Milan, Italy}
\item Conducting analysis for a multivariate twin study with R (OpenMx)
\item Data wrangling with SPSS and R% that's the most accurate term, =/= data cleansing
\item Preparing tables and reports in R and Microsoft Word
\item Checking grammar, punctuation, and style in written academic English
\end{rSubsection}

\begin{rSubsection}{Research Assistant}{July 2017}{Sigmund Freud University, Milan}{Milan, Italy}
\item Data wrangling in shell script for EEG twin data analysis pipeline
\item Data cleansing and proof-reading codebook norms
\end{rSubsection}

\begin{rSubsection}{Research Assistant--Translator (English to Italian)}{June 2016}{Sigmund Freud University, Milan}{Milan, Italy}
\item Back-translating a psychological questionnaire for children and young adolescents
\item Choosing sensible phraseology in the target language
\end{rSubsection}

\end{rSection}

%----------------------------------------------------------------------------------------

\begin{rSection}{Professional Development}

% I'm going to be honest, I don't remember the difference between rSubsection and rNSubsection. But this is how I formatted things that don't have bullet points beneath them.

% \begin{rNSubsection}{Course or Certification Name}{Dates}{Company/Organization}{Location}
% \item
% \end{rNSubsection}

% \vspace{-7mm} % The \vspace{} is necessary to get rid of weird spacing that I haven't figured out how to fix yet... ¯\_(ツ)_/¯

\begin{rNSubsection}{Mendelian Randomisation workshop}{April 2020}{European Mathematical Genetics Meeting}{Lausanne, Switzerland}
\item
\end{rNSubsection}

\vspace{-7mm}

\begin{rNSubsection}{Prediction modelling course}{December 2018}{Department for Biostatistics and Health Informatics, King's College London}{London, UK}
\item
\end{rNSubsection}

\vspace{-7mm}

\begin{rNSubsection}{Meta-analysis course}{November 2018}{
Department for Biostatistics and Health Informatics, King's College London}{London, UK}
\item
\end{rNSubsection}

\vspace{-7mm}

\end{rSection}

\pagebreak % I have to add this or it's super ugly :(
%----------------------------------------------------------------------------------------

% \begin{rSection}{Related Professional Skills}

% % This is how I've been doing bulleted lists of professional skills.

% \vspace{-3mm}

% \begin{rSubsection}{}{}{}{}
	% \item OS: Microsoft Windows, GNU/Linux
	% \item King's College London Rosalind {}HPC research infrastructure
	% \item Programming languages: R, Python, scripting language (bash, SSH)
	% \item Statistical analysis: R, Excel, SPSS, Stata
	% \item Productivity Tools: Microsoft Office Suite, \LaTeX \hspace{.1mm} Typesetting Environment, Emacs, Zoom
	% \item Collaboration Tools: GitHub, Dropbox, Google Drive
% \end{rSubsection}

% \end{rSection}

% If you want two columns of skills, you could also comment out the above section and use the code below, which uses the multicol package. It all depends on how you want it to look!

\begin{rSection}{Related Professional Skills}
\begin{multicols}{2}[]
  \begin{itemize}
    \item OS: \@ Microsoft Windows, GNU/Linux, Mac OS X
    \item King's College London Rosalind HPC research infrastructure
    \item Coding: \@ R, Python, scripting language (bash, SSH)
    \item Statistical analysis: \@ R, Microsoft Excel, SPSS, Stata
    \item Statistical genetics software: \@ PLINK, Haploview, PRSice
	\item Productivity Tools: \@ Microsoft Office Suite, LibreOffice suite, \LaTeX \hspace{.1mm} Typesetting Environment, Emacs, Zoom, Zotero
    \item Collaboration Tools: \@ GitHub, Dropbox, Google Drive
    \item Languages: \@ Italian (first language), English (IELTS overall band score: 7.5, 30/06/18), Lombard language, Latin (translation only), Ancient Greek (translation only), French (basic)
\end{itemize}
\end{multicols}
\end{rSection}

%----------------------------------------------------------------------------------------

\end{document}
